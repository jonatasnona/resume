%
% This work may be distributed and/or modified under the
% conditions of the LaTeX Project Public License version 1.3c,
% available at http://www.latex-project.org/lppl/.


\documentclass[11pt,a4paper]{moderncv}

% moderncv
% optional colors: black, blue, green, grey, orange, purple and red
% optional layouts: banking, casual, classic and oldstyle
\moderncvtheme[blue]{classic}

% character encoding
% set when using with pdflatex
%\usepackage[utf8]{inputenc}

\usepackage{textcomp}
\usepackage{amssymb}
\usepackage{enumitem}

% adjust the page margins
\usepackage[scale=0.8]{geometry}

% set when using with lualatex
% unset when using with pdflatex
\usepackage{fontawesome}
\usepackage{moderncviconsawesome}

% ubuntu font
%\usepackage{ubuntu}

\renewcommand{\familydefault}{\sfdefault}
\setitemize[0]{leftmargin=10pt,itemindent=10pt}

% personal data
\firstname{Jonatas}
\familyname{Pedraza}
\title{Software Developer}
\address{Estrada de Jacarepagua 4430 - Anil, 208}{Rio de Janeiro/RJ, Brasil}
\mobile{(+55) 21 99971-3374}
\email{jonatas.nona@gmail.com}
\social[github]{jonatasnona}
\social[twitter]{@jonatasnona}
\social[linkedin][br.linkedin.com/pub/jonatas-pedraza/8/959/91]{Jonatas Pedraza}
%\photo[64pt]{data/avatar.jpg}

% to show numerical labels in the bibliography; only useful if you make citations in your resume
\makeatletter
\renewcommand*{\bibliographyitemlabel}{\@biblabel{\arabic{enumiv}}}
\makeatother

%----------------------------------------------------------------------------------
%            content
%----------------------------------------------------------------------------------
\begin{document}
\maketitle

\section{Experiência}
\cventry{Atualmente}{Site Reliability Engineer}{Globo.com}{Rio de Janeiro/RJ, Brasil}{}{
\vspace{1pt} \textbf{Departamento:} Webmedia
\newline
\newline
\textbf{Atividades:}
    \begin{itemize}
        \item Desenvolvimento de serviços backend.
        \item Desenvolvimento de REST APIs.
        \item Desenvolvimento Frontend.
        \item Git Reviewer.
    \end{itemize}
\textbf{Projetos:}
    \begin{itemize}
        \item Person Of Interest (POI) - serviço que detecta em real time qual stream (programa) ao vivo uma personalidade de interesse apareceu e notifica sua aparição usando algoritmos de face detection.
        \item Thumbs Analyzer - serviço de checagem da feature de dvr da plataforma de vídeos ao vivo.
        \item Placeholder Recognizer - serviço que checa se um dado stream (programa) é um tapume ou video congelado usando similaridade de imagens.
        \item Browser Report - serviço que faz um report completo do browser para análises de problemas de clientes que não conseguem tocar vídeos da globo.com.
    \end{itemize}
\vspace{1pt}\textbf{Tecnologias: NodeJS, Python, MongoDB, Redis, Git, Tsuru, Docker, Bootstrap}
\hspace{1pt}
\newline{}}

\cventry{2013-2014}{Software Developer}{Neemu}{Manaus/AM, Brasil}{}{
\vspace{1pt} \textbf{Departamento:} Big Data
\newline
\newline
\textbf{Atividades:}
    \begin{itemize}
        \item Arquitetura de sistemas distribuídos.
        \item Desenvolvimento de REST APIs.
        \item Desenvolvimento de serviços de aquisição de dados.
        \item Desenvolvimento de serviços backend.
        \item Desenvolvimento de unittests para validação de documentos JSON.
        \item Modelagem de schemas no MongoDB.
        \item Git Reviewer.
    \end{itemize}
\textbf{Projetos:}
    \begin{itemize}
        \item Cockpit Dashboard - web dashboard para os clientes da Neemu. Com essa ferramenta os clientes da Neemu conseguem acompanhar a performance dos sistemas de busca e recomendação da Neemu através de uma visualização gráfica.
            Esta experiência permite aos clientes identificar no seu site em pouco tempo quais são os produtos mais vendidos, os mais procurados, a melhor regra de performance para o sistema de recomendação, a performance do autocomplete de recomendação dentre outras informações.
            Com estas métricas o cliente consegue tomar as melhores decisões para alavancar a performance do seu website.
        \item SearchManager - sistema que permite tunar o sistema de busca para obter melhores resultados e priorizar um produto específico.
    \end{itemize}
\vspace{1pt}\textbf{Tecnologias: NodeJS, Python, MySQL, MongoDB, Redis, Git}
\hspace{1pt}
\newline{}}

\cventry{}{Software Developer}{Neemu}{Manaus/AM, Brasil}{}{
\vspace{1pt} \textbf{Departamento:} Infraestrutura
\newline
\newline
\textbf{Atividades:}
    \begin{itemize}
        \item Criação de rotinas de deploy.
        \item Gerenciamento e manutenção dos servidores na cloud da Amazon e Rackspace.
        \item Rotinas de backup.
        \item Mantenedor do servidor Git (Gitlab) interno.
    \end{itemize}
\vspace{1pt}\textbf{Tecnologias: Python, Fabric, Git}
\hspace{1pt}
\newline{}}

\cventry{}{Software Developer}{Neemu}{Manaus/AM, Brasil}{}{
\vspace{1pt} \textbf{Departamento:} Enriquecimento
\newline
\newline
\textbf{Atividades:}
    \begin{itemize}
        \item Desenvolvimento de web crawlers.
        \item Desenvolvimento Frontend.
        \item Arquitetura de microservices.
        \item Extração de texto.
        \item Manutenção de código legado.
        \item Git Reviewer.
    \end{itemize}
\textbf{Projetos:}
    \begin{itemize}
        \item Fetcher Multithreaded - serviço super rápido para coleta de imagens e páginas html para uso do time de desenvolvimento.
        \item Distributed Data Collection - remodelagem do serviço de coleta e extração de texto para uma abordagem modular e distribuída usando microservices.
        \item Neemu Catalog - serviço para coletar e catalogar as lojas parceiras e mostrar seus produtos com descrições completas e preços.
    \end{itemize}
\vspace{1pt}\textbf{Tecnologias: Java, Python, NodeJS, Bootstrap, MongoDB}
\hspace{1pt}
\newline{}}

\cventry{2012--2013}{Software Developer}{INdT}{Manaus/AM, Brasil}{}{
\vspace{1pt} \textbf{Departamento:} Cloud \& Backend
\newline
\newline
\textbf{Atividades:}
    \begin{itemize}
        \item Desenvolvimento de web apps para a plataforma S40 da Nokia.
        \item Desenvolvimento de apps para Windows Phone.
        \item Desenvolvimento de serviços backend.
        \item Cursos sobre boas práticas de desenvolvimento de software.
        \item Git Reviewer.
        \item Empacotador Debian e RPM.
    \end{itemize}
\textbf{Projetos:}
    \begin{itemize}
        \item SaveMe - aplicativo de compras coletivas.
        \item FolderIdea - aplicativo padrão de uma operadora de telefonia celular mexicana.
        \item Manatus - Nokia CTO
    \end{itemize}
\vspace{1pt}\textbf{Tecnologias: C\#, Java, JavaScript, NodeJS, Git}
\hspace{1pt}
\newline{}}

\cventry{}{Test Analyst}{INdT}{Manaus/AM, Brasil}{}{
\vspace{1pt} \textbf{Departamento:} Product Creation
\newline
\newline
\textbf{Atividades:}
    \begin{itemize}
        \item Automação de testes.
        \item Build releases.
    \end{itemize}
\textbf{Projetos:}
    \begin{itemize}
        \item PagSeguro NFC Payment - aplicativo do PagSeguro para realizar transações através de NFC.
        \item PagSeguro Dummy REST API - serviço que emulava a API REST do PagSeguro.
    \end{itemize}
\vspace{1pt}\textbf{Tecnologias: Java, Qt, Selenium, Git}
\hspace{1pt}
\newline{}}

\cventry{2010--2012}{Scholarship Researcher}{INPA}{Manaus/AM, Brasil}{}{
\vspace{1pt} \textbf{Departamento:} Núcleo de BioGeo Informática (NBGI)
\newline
\newline
\textbf{Atividades:}
    \begin{itemize}
        \item Desenvolvimento de sistemas internos e estudos aplicados em Web Sem\^{a}ntica.
        \item Gerenciamento da farm de servidores Linux.
        \item Gerenciamento e manuntenção do servidor Git.
    \end{itemize}
\textbf{Projetos:}
    \begin{itemize}
        \item Ornitólogo Automático - serviço web desenvolvido com OpenLayers para visualização de espécies de aves localizadas na Amazônia legal de acordo com os dados obtidos de um banco de dados geoespacial.
        \item SisProj - site de compartilhamento e interação de artigos e projetos desenvolvidos pelos pesquisadores do projeto LBA.
    \end{itemize}
\vspace{1pt}\textbf{Tecnologias: PHP, MySQL, Postgres, Git}
\hspace{1pt}
\newline{}}

\cventry{2008--2010}{Scholarship Intern}{FUCAPI}{Manaus/AM, Brasil}{}{
\vspace{1pt} \textbf{Departamento:} Zagaia
\newline
\newline
\textbf{Atividades:}
    \begin{itemize}
        \item Desenvolvimento de projetos embarcados para dispositivos Nokia.
        \item Empacotador oficial.
    \end{itemize}
\textbf{Projetos:}
    \begin{itemize}
        \item PC Remote - uma aplicação desenvolvida em Python que permitia a interação do teclado e mouse a partir de um device N810 com o desktop.
        \item KFluid - um plasmoide KDE desenvolvido em Qt responsável por realizar transferências de arquivos via bluetooth através de drag and drop.
    \end{itemize}
\vspace{1pt}\textbf{Tecnologias: Python, C, Qt, Git}
\hspace{1pt}
\newline{}}

\section{Cursos}
\cventry{2013}{\faOk \hspace{1pt} MongoDB for NodeJS Developers}{}{\href{https://s3.amazonaws.com/edu-cert.10gen.com/downloads/2d484817850246ef8c69e4f7fd0c8fa0/Certificate.pdf}{MongoDB University}}{}{Online course lectured by Andrew Erlichson.\newline{}}
\cventry{2012}{\faOk \hspace{1pt} Try jQuery}{}{\href{http://www.codeschool.com/users/jonatasnona}{CodeSchool}}{}{Online course lectured by Gregg Pollack.\newline{}}
\cventry{}{\faOk \hspace{1pt} Try Git}{}{\href{http://www.codeschool.com/users/jonatasnona}{CodeSchool}}{}{Online course lectured by Gregg Pollack.\newline{}}
\cventry{}{\faOk \hspace{1pt} jQuery Air: First Flight}{}{CodeSchool}{}{Online course offered by \href{http://www.codeschool.com/users/jonatasnona}{CodeSchool}.\newline{}}
\cventry{}{\faOk \hspace{1pt} 30 Days to Learn jQuery}{}{Nettuts+}{}{Online course lectured by Jeffrey Way.\newline{}}

\section{Idiomas}
\cvlanguage{Inglês}{Intermediário}{}

\section{Skills}
\cvline{OS}{Linux$\bigstar$, Debian Packaging}
\cvline{Programming}{NodeJS$\bigstar$, Python, Java, C}
\cvline{Frameworks}{jQuery$\bigstar$, Bootstrap}
\cvline{Databases}{Redis, MongoDB, MySQL}
\cvline{VCS}{Git$\bigstar$}
\cvline{Toolbelt}{Docker, Vagrant, Automake, Bower, Fabric, Maven}
\cvline{IDE}{VIM$\bigstar$}
\cvline{Automation}{Selenium, PhantomJS, CasperJS}

\section{Publicações}
\cvitem{2010}{Pedraza Jônatas, Campos Laurindo, \faTrophy \hspace{1pt} \textit{``Criação e Mapeamento
de Ontologias de Domínio de Biodiversidade''}, II Escola Regional de Informática. Manaus/AM, Brasil.}
\cvitem{2009}{Bilby Henry, Junior Antônio, Oliveira Samuel, Pedraza Jônatas,
Portela André, \textit{``Desenvolvimento de Sistemas Pervasivos com Bluetooth e
Linux/Python''}, 10\textordmasculine \ Fórum Internacional Software Livre. Porto Alegre/RS, Brasil.}

\section{Palestras e Minicursos}
\cvitem{2010}{\textit{``Ubuntu: Como instalar e usar''}, FLISOL. Manaus/AM, Brasil.}

\section{Prêmios Acadêmicos}
\cvlistitem{``Prêmio de Melhor Pôster'' em 2010, pelo evento II Escola Regional de Informática.}

\section{Leituras}
\cvitem{\faBookmark}{\textit{``Node.js The Right Way''}, 2013 Jim R. Wilson. The Pragmatic Programmers.}
\cvitem{\faBookmark}{\textit{``REST API Design Rulebook''}, 2012 Mark Massé. O'REILLY.}
\cvitem{\faBookmark}{\textit{``PHP Programando com Orientação a Objetos''}, 2007 Pablo Dall'Oglio. Novatec.}

%\section{Interests}
%\cvline{Photography}{\small Amateur photographer}
%\cvline{Science-Fiction}{\small Reader of science-fiction novels.}
%\cvline{Music}{\small Fond of alternative and indie music.}

\end{document}
