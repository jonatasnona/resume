%
% This work may be distributed and/or modified under the
% conditions of the LaTeX Project Public License version 1.3c,
% available at http://www.latex-project.org/lppl/.


\documentclass[11pt,a4paper]{moderncv}

% moderncv
% optional colors: black, blue, green, grey, orange, purple and red
% optional layouts: banking, casual, classic and oldstyle
\moderncvtheme[blue]{classic}

% character encoding
% set when using with pdflatex
%\usepackage[utf8]{inputenc}

\usepackage{textcomp}
\usepackage{amssymb}

% adjust the page margins
\usepackage[scale=0.8]{geometry}

% set when using with lualatex
% unset when using with pdflatex
\usepackage{fontawesome}
\usepackage{moderncviconsawesome}

% ubuntu font
%\usepackage{ubuntu}

\renewcommand{\familydefault}{\sfdefault}

% personal data
\firstname{Jonatas}
\familyname{Pedraza}
\title{Software Developer}
\address{Estrada de Jacarepagua 4430 - Anil, 208}{Rio de Janeiro/RJ, Brasil}
\mobile{(+55) 21 99971-3374}
\email{jonatas.nona@gmail.com}
\social[github]{jonatasnona}
\social[twitter]{@jonatasnona}
\social[linkedin][br.linkedin.com/pub/jonatas-pedraza/8/959/91]{Jonatas Pedraza}
%\photo[64pt]{data/avatar.jpg}

% to show numerical labels in the bibliography; only useful if you make citations in your resume
\makeatletter
\renewcommand*{\bibliographyitemlabel}{\@biblabel{\arabic{enumiv}}}
\makeatother

%----------------------------------------------------------------------------------
%            content
%----------------------------------------------------------------------------------
\begin{document}
\maketitle

\section{Experiência}
\cventry{Atualmente}{Site Reliability Engineer - Webmedia}{Globo.com}{Rio de Janeiro/RJ, Brasil}{}{}

\cventry{2013-2014}{Software Developer - Big Data}{Neemu}{Manaus/AM, Brasil}{}{
Desenvolvedor backend do produto \textbf{Cockpit Dashboard}.
Criação de unittests para validação de documentos JSON. Modelagem de schemas no MongoDB.
Criação de serviços para aquisição de dados em NodeJS.
Participação no desenvolvimento do protótipo \textbf{Monitor SEO}, desenvolvido com Python e MongoDB.
\newline{}}

\cventry{}{Software Developer - Infraestrutura}{Neemu}{Manaus/AM, Brasil}{}{
Gerenciamento de servidores na cloud da Amazon AWS, Rackspace e Linode.
Gerenciamento do servidor de respositórios GIT. Criação de rotinas de deploy
dos principais serviços. Monitoria com o Nagios.
\newline{}}

\cventry{}{Software Developer - Frontend}{Neemu}{Manaus/AM, Brasil}{}{
Participação no desenvolvimento e arquitetura de uma API REST para o produto \textbf{Cockpit Dashboard} desenvolvido com NodeJS, Python, MySQL e Redis.
\newline{}}

\cventry{}{Software Developer - Enriquecimento}{Neemu}{Manaus/AM, Brasil}{}{
Criação de um Fetcher Multithreaded e customizável desenvolvido em Java exclusivo para coleta de
imagens e páginas html para uso interno dos desenvolvedores da Neemu.
Desenvolvimento de WebCrawlers e WebSpiders. Participação ativa na arquitetura
e desenvolvimento do novo coletor modularizado e multithreaded da Neemu com troca de mensagens
entre os módulos usando TCP Sockets, desenvolvidos com Java e NodeJS.
Desenvolvimento frontend e backend do protótipo \textbf{Neemu Catalog}, serviço para catalogar
as lojas parceiras e mostrar seus produtos com descrições completas e seus preços,
desenvolvido com Bootstrap, NodeJS e MongoDB.
\newline{}}

\cventry{2012--2013}{Software Developer - Cloud \& Backend}{INdT}{Manaus/AM, Brasil}{}{
Desenvolvedor do aplicativo de compras coletivas SaveMe para a plataforma Nokia S40.
Desenvolvimento de 'template engine' em C\# usando HubTiles para o aplicativo FolderIdea.
Desenvolvedor ativo do projeto Manatus (Nokia CTO), implentação de REST APIs
com NodeJS, git reviewer e code integration com times remotos, empacotador debian e rpm.
Frontend Group Member - profissionais do time Cloud \& Backend com expertise em frontend, responsáveis por compartilhar boas práticas de
desenvolvimento das páginas web, sugerir ferramentas de performance e criar componentes de UI
para suprir necessidades de outros times.
\newline{}}

\cventry{}{Test Analyst - Product Creation}{INdT}{Manaus/AM, Brasil}{}{
Contrato temporário para compor o time de Product Creation no projeto PagSeguro NFC
Payment. Responsável por realizar baterias de testes para validar a integridade
e funcionalidade do aplicativo. Gerar builds de release ao final de cada
sprint. Manter o servidor dummy do pagseguro ativo e implementar métodos
da REST API no ambiente de desenvolvimento.
\newline{}}

\cventry{2010--2012}{Scholarship Researcher}{INPA}{Manaus/AM, Brasil}{}{
Desenvolvedor do Ornitólogo Automático, serviço web desenvolvido com OpenLayers para visualização de
espécies de aves localizadas na Amazônia legal de acordo com os dados obtidos de um banco de dados geoespacial.
Desenvolvedor do SisProj, site de compartilhamento e interação de artigos e projetos desenvolvidos pelos pesquisadores do projeto LBA.
Realização de pesquisas com Web Sem\^{a}ntica, administração e manutenção do
parque de servidores Linux e manutenção do servidor de repositórios GIT
do NBGI.
\newline{}}

\cventry{2008--2010}{Scholarship Intern}{FUCAPI}{Manaus/AM, Brasil}{}{
Bolsista do projeto Zagaia, projeto realizado a partir de uma parceria com a
FUCAPI e o INdT com o objetivo de desenvolver software embarcado para dispositivos Nokia.
Desenvolvedor do PC REMOTE, uma aplicação escrita em Python para o N810 que permitia a interação
do teclado e mouse a partir de um device N810 com o desktop.
Desenvolvedor do KFluid, um plasmoid KDE desenvolvido em Qt responsável por realizar
transferências de arquivos via bluetooth através de drag and drop.
Empacotador debian oficial dos projetos.
\newline{}}

\section{Cursos}
\cventry{2013}{\faOk \hspace{1pt} MongoDB for NodeJS Developers}{}{\href{https://s3.amazonaws.com/edu-cert.10gen.com/downloads/2d484817850246ef8c69e4f7fd0c8fa0/Certificate.pdf}{MongoDB University}}{}{Online course lectured by Andrew Erlichson.\newline{}}
\cventry{2012}{\faOk \hspace{1pt} Try jQuery}{}{\href{http://www.codeschool.com/users/jonatasnona}{CodeSchool}}{}{Online course lectured by Gregg Pollack.\newline{}}
\cventry{}{\faOk \hspace{1pt} Try Git}{}{\href{http://www.codeschool.com/users/jonatasnona}{CodeSchool}}{}{Online course lectured by Gregg Pollack.\newline{}}
\cventry{}{\faOk \hspace{1pt} jQuery Air: First Flight}{}{CodeSchool}{}{Online course offered by \href{http://www.codeschool.com/users/jonatasnona}{CodeSchool}.\newline{}}
\cventry{}{\faOk \hspace{1pt} 30 Days to Learn jQuery}{}{Nettuts+}{}{Online course lectured by Jeffrey Way.\newline{}}

\section{Idiomas}
\cvlanguage{Inglês}{Intermediário}{}

\section{Skills}
\cvline{SO}{Linux$\bigstar$, Debian Packaging}
\cvline{Linguagens de Programação}{C, Java, PHP, Python, JavaScript\faStarHalfEmpty, Shell script}
\cvline{Linguagens de Marcação}{YAML, XML, LaTeX, CSS3, HTML5, Markdown}
\cvline{Frameworks}{jQuery$\bigstar$, Bootstrap}
\cvline{Databases}{MySQL, PostgreSQL, Redis}
\cvline{Sistemas de Controle de Versão}{Git$\bigstar$, SVN}
\cvline{Ferramentas de Build e Deploy}{Automake, Autoconf, ant, Maven, Bower,
Jenkins, Fabric}
\cvline{CMS}{Wordpress}
\cvline{IDE}{Eclipse, VIM$\bigstar$}
\cvline{Automação de Testes}{Selenium, PhantomJS, CasperJS}
\cvline{Bugtracker}{JIRA}
\cvline{Featured Technologies}{NodeJS$\bigstar$, MongoDB}

\section{Publicações}
\cvitem{2010}{Pedraza Jônatas, Campos Laurindo, \faTrophy \hspace{1pt} \textit{``Criação e Mapeamento
de Ontologias de Domínio de Biodiversidade''}, II Escola Regional de Informática. Manaus/AM, Brasil.}
\cvitem{2009}{Bilby Henry, Junior Antônio, Oliveira Samuel, Pedraza Jônatas,
Portela André, \textit{``Desenvolvimento de Sistemas Pervasivos com Bluetooth e
Linux/Python''}, 10\textordmasculine \ Fórum Internacional Software Livre. Porto Alegre/RS, Brasil.}

\section{Palestras e Minicursos}
\cvitem{2010}{\textit{``Ubuntu: Como instalar e usar''}, FLISOL. Manaus/AM, Brasil.}

\section{Prêmios Acadêmicos}
\cvlistitem{``Prêmio de Melhor Pôster'' em 2010, pelo evento II Escola Regional de Informática.}

\section{Leituras}
\cvitem{\faBookmark}{\textit{``Node.js The Right Way''}, 2013 Jim R. Wilson. The Pragmatic Programmers.}
\cvitem{\faBookmark}{\textit{``REST API Design Rulebook''}, 2012 Mark Massé. O'REILLY.}
\cvitem{\faBookmark}{\textit{``PHP Programando com Orientação a Objetos''}, 2007 Pablo Dall'Oglio. Novatec.}

%\section{Interests}
%\cvline{Photography}{\small Amateur photographer}
%\cvline{Science-Fiction}{\small Reader of science-fiction novels.}
%\cvline{Music}{\small Fond of alternative and indie music.}

\end{document}
